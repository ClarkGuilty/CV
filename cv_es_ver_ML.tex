\documentclass[11pt, a4paper]{article}
%\usepackage{fontspec} 
\usepackage{etaremune}
\usepackage{csquotes}
% DOCUMENT LAYOUT
\usepackage{geometry} 
\geometry{a4paper, textwidth=6.1in, textheight=10.2in, marginparsep=7pt, marginparwidth=.6in}
\setlength\parindent{0in}


\usepackage[spanish]{babel}
\usepackage[bookmarks, colorlinks, breaklinks, 
% ---- FILL IN HERE THE TITLE AND AUTHOR
	pdftitle={Javier Acevedo - vita},
	pdfauthor={Javier Acevedo},
]{hyperref}

\usepackage[utf8]{inputenc} % Acepta caracteres en castellano
\usepackage[T1]{fontenc} % Encoding de salida al pdf

\usepackage{marginnote}
\newcommand{\years}[1]{\marginnote{\scriptsize #1}}
\renewcommand*{\raggedleftmarginnote}{}
\setlength{\marginparsep}{7pt}


% ---- CUSTOM COMMANDS

\newcommand{\html}[1]{\href{#1}{\scriptsize\textsc{[html]}}}
\newcommand{\pdf}[1]{\href{#1}{\scriptsize\textsc{[pdf]}}}
\newcommand{\doi}[1]{\href{#1}{\scriptsize\textsc{[doi]}}}

\renewcommand*{\raggedleftmarginnote}{}
\setlength{\marginparsep}{7pt}
\reversemarginpar









\begin{document}
\begin{center}{\huge \bf Javier Alejandro Acevedo Barroso}\\[1cm]\end{center}
\begin{minipage}[t]{0.465\textwidth}
  Teléfono: (+57) 301-680-9844 \\
  Email: \href{mailto:ja.acevedo12@uniandes.edu.co}{ja.acevedo12@uniandes.edu.co}\\
  Email: \href{mailto:ja.acevedob12@gmail.com}{ja.acevedob12@gmail.com}\\
  Linkedin: \href{https://www.linkedin.com/in/javier-alejandro-acevedo-barroso-494155203/}{Perfil}\\
  Github: \href{https://github.com/clarkguilty}{Perfil}\\
\end{minipage}

\hrule

\section*{Información Personal}
Nacido en Bucaramanga, Colombia, el 4 de enero de 1997 (24 años).

\section*{Áreas de Interés}
\begin{itemize}
\item Machine Learning e inteligencia artifical aplicada.
\item Clasificación de series de tiempo.
\item Desarrollo de modelos predictivos.
\item Visualización eficiente de datos
\item Reconstrucción y modelamiento de lentes gravitacionales.
\item Búsqueda de estrellas variables.
\item Cosmología observacional.

%\item Filosofía de la pedagogía.
%\item Materia oscura.

\end{itemize}

\section*{Educación}
\noindent
\years{2015-2017}\textsc{Cuatro semestres de pregrado en Ingeniería de sistemas}\\ {\emph{Institución}}: Departamento de Ingeniería de Sistemas y Computación, Universidad de los Andes.\\

\years{2015-2019}\textsc{Pregrado en Física}\\ {\emph{Institución}}: Departamento de Física, Universidad de los Andes. {\emph{Tesis}}: Simulación de materia oscura colisional con un método de Lattice-Boltzmann. {\emph{Director}}: Dr. Jaime Forero.\\

\years{2019-2021}\textsc{Maestría en Ciencias-Física}\\ {\emph{Institución}}: Departamento de Física, Universidad de los Andes. {\emph{Tesis}}: Búsqueda de estrellas variables extragalácticas usando algoritmos de Machine Learning. {\emph{Director}}: Dr. Alejandro García.\\
\section*{Participación en Eventos}
\years{2018}{Escuela de Astronomía Uniandes 2018.}\\ {\emph{Institución}}: Departamento de Física, Universidad de los Andes. \\
\years{2018}{MOCa 2018: Materia Oscura en Colombia}\\ {\emph{Institución}}: Departamento de Física, Universidad de los Andes. {\emph{Charla}}: Simulating Collisional Dark Matter.\\
\years{2019}{MOCa 2019: Materia Oscura en Colombia}\\ {\emph{Institución}}: Departamento de Física, Universidad de los Andes. {\emph{Charla}}: Simulating collisional dark matter.\\
\years{2019}{COCOA 2019 Medellín: VI Congreso Colombiano de Astronomía y Astrofísica }\\ {\emph{Organizadores}}: Universidad de Antioquia, Parque Explora – Planetario de Medellín, Instituto Tecnológico Metropolitano ITM y Sociedad Antioqueña de Astronomía SAA. {\emph{Charla}}: Simulando materia oscura colisional.\\
\section*{Actividad de Investigación}
\years{2019-En curso}{Modelamiento de lente gravitacional usando datos del telescopio de 2.2-m ESO/MPG para medir $H_0$ (H0LICOW)}\\ {\emph{Institución}}: Departamento de Física, Universidad de los Andes. {\emph{Director}}: Dr. Alejandro García y Dr. Frédéric Courbin.\\	
\years{2019-2021}{Búsqueda de estrellas variables extragalácticas usando algoritmos de Machine Learning}\\ {\emph{Institución}}: Departamento de Física, Universidad de los Andes. {\emph{Director}}: Dr. Alejandro García.\\	
\years{2019}{Medición de la velocidad de rotación de estrellas tipo B y A}\\ {\emph{Institución}}: Departamento de Física, Universidad de los Andes. {\emph{Director}}: Dr. Alejandro García.\\	
\years{2018-2020}{Simulación de materia oscura colisional con un método de Lattice-Boltzmann.}\\ {\emph{Institución}}: Departamento de Física, Universidad de los Andes. {\emph{Director}}: Dr. Jaime Forero.\\	
\years{2017}{Caracterización de materiales utilizando tomografía de Muones}\\ {\emph{Institución}}: Departamento de Física, Universidad de los Andes. {\emph{Director}}: Dr. Carlos Ávila.\\

\section*{Experiencia Docente}
\years{2019-2020}{Asistente graduado, Física experimental I.}\\ {\emph{Institución}}: Departamento de Física, Universidad de los Andes. {\emph{Supervisor}}: Germán Andrés Moreno Cely.\\	
\years{2019-2020}{Asistente graduado, Física experimental II.}\\ {\emph{Institución}}: Departamento de Física, Universidad de los Andes. {\emph{Supervisor}}: Germán Andrés Moreno Cely.\\	
\years{2017-2018}{Tutor de la Clínica de Problemas de Física.}\\ {\emph{Institución}}: Departamento de Física, Universidad de los Andes. {\emph{Supervisor}}: Juan Diego Arango Montoya.\\	

\section*{Otro trabajos y cursos}
\years{2019}{Diagramación del libro \textquote{Las Bolsas de Basura} de Enrique Winter.}\\ {\emph{Editorial}}: Escarabajo editorial.\\	
\years{2020}{Data-Driven Astronomy. Coursera: The University of Sydney.}\\ 
\years{2020}{Support Vector Machines with scikit-learn Coursera: Coursera Project Network.}\\ 

\section*{Reconocimientos y Becas}
\years{2019}{Reconocimiento a mejores puntajes. Prueba Saber Pro 2018. Otorgado por el Ministerio de Educación Nacional.}\\ 
\years{2019}{Asistencia graduada para maestría en Ciencias-Física con beca completa, Universidad de los Andes.\\
\years{2014}{Beca Bachilleres por Colombia, Programa Mario Galán Gómez }. Otorgada por Ecopetrol.\\
\years{2013}{Mejor estudiande del departamento de Santander. Prueba Saber 11 2013. Otorgado por el Ministerio de Educación Nacional.}\\ 

\section*{Habilidades Adicionales}
\begin{itemize}
%\item Alta habilidad docente y pasión por enseñar.
\item Facilidad de aprendizaje.
\item Enseñanza de fudamentos de la física y matemática universitarias.
\item Pedagogía virtual, incluyendo diseño y dictado de cursos de laboratorio.
\item Alta capacidad de resolución de problemas.
\item Plataforma Blackboard, y herramientas para desarrollo de clases virtuales (Zoom, Teams, Discord).
\item Atención al detalle.
\item Buen trabajo en equipo, y autosuficiencia en trabajo individual.
\item Lenguajes: Español (nativo), Inglés (C1), Alemán (A1).
\item Sistemas operativos Linux y Windows. 
\item Diagramación profesional en \LaTeX.
\item Lenguajes de programación: C/C++, Python, R, Java, Bash, Julia.
\item SQL básico en SQlite y PySQL.
\item Habilidades de ofimática.
\item SSH y protocolos asociados.
\item Paralelización con MPI y OpenMP.
\item Manejo de telescopio y reducción de datos astronómicos.
\item Implementación de soluciones con inteligencia artificial (PyTorch, Tensorflow, Keras, Flux).
\item Escritura científica.
\item Automatización de tareas.
\item Visualización de datos con Dash, Seaborn, Matplotlib y Gnuplot.
\item Electrónica básica y manejo de Arduino.
\item Simulaciones computaciones y métodos numéricos.
\item Uso de software adicional: Awk, Anaconda, IRAF, Sympy, Git, Maxima, Optuna, Scikit-learn, Vim.
\end{itemize}

\section*{Referencias}
\begin{itemize}
\item Dr. Jose Alejandro Garcia Varela\\
Profesor Departamento de física Universidad de los Andes.\\
Email: \href{mailto:josegarc@uniandes.edu.co}{josegarc@uniandes.edu.co}

\item Dr. Jaime Ernesto Forero Romero\\
Profesor Departamento de física Universidad de los Andes.\\
Email: \href{mailto:je.forero@uniandes.edu.co}{je.forero@uniandes.edu.co}

\item Dr. Beatriz Eugenia Sabogal Martinez\\
Profesora Departamento de física Universidad de los Andes.\\
Email: \href{mailto:bsabogal@uniandes.edu.co}{bsabogal@uniandes.edu.co}
\end{itemize}

\end{document}
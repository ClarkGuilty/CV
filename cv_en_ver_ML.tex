\documentclass[11pt, a4paper]{article}
%\usepackage{fontspec} 
\usepackage{etaremune}

% DOCUMENT LAYOUT
\usepackage{geometry} 
\geometry{a4paper, textwidth=6.1in, textheight=10.2in, marginparsep=7pt, marginparwidth=.6in}
\setlength\parindent{0in}

\usepackage[english]{babel}
\usepackage[bookmarks, colorlinks, breaklinks, 
% ---- FILL IN HERE THE TITLE AND AUTHOR
	pdftitle={Javier Acevedo - vita},
	pdfauthor={Javier Acevedo},
]{hyperref}

\usepackage[utf8]{inputenc} % Acepta caracteres en castellano
\usepackage[T1]{fontenc} % Encoding de salida al pdf

\usepackage{marginnote}
\newcommand{\years}[1]{\marginnote{\scriptsize #1}}
\renewcommand*{\raggedleftmarginnote}{}
\setlength{\marginparsep}{7pt}


% ---- CUSTOM COMMANDS

\newcommand{\html}[1]{\href{#1}{\scriptsize\textsc{[html]}}}
\newcommand{\pdf}[1]{\href{#1}{\scriptsize\textsc{[pdf]}}}
\newcommand{\doi}[1]{\href{#1}{\scriptsize\textsc{[doi]}}}

\renewcommand*{\raggedleftmarginnote}{}
\setlength{\marginparsep}{7pt}
\reversemarginpar









\begin{document}
\begin{center}{\huge \bf Javier Alejandro Acevedo Barroso}\\[1cm]\end{center}
\begin{minipage}[t]{0.465\textwidth}
  Phone: (+57) 301-680-9844 \\
  Email: \href{mailto:ja.acevedo12@uniandes.edu.co}{ja.acevedo12@uniandes.edu.co}\\
  Email: \href{mailto:ja.acevedob12@gmail.com}{ja.acevedob12@gmail.com}\\
  Linkedin: \href{https://www.linkedin.com/in/javier-alejandro-acevedo-barroso-494155203/}{Profile}\\
  Github: \href{https://github.com/clarkguilty}{Profile}\\
\end{minipage}

\hrule

\section*{Personal information}
Born in Bucaramanga, Colombia, on January 4\textsuperscript{th}, 1997. Citizenship: Colombian.

\section*{Interest areas}
\begin{itemize}
%\item Reconstrucción y modelamiento de lentes gravitacionales fuertes.
\item Machine learning and artificial intelligence
\item Time series prediction and classification.
\item Machine Learning in astronomy.
\item Predictive models.
\item Data visualization.
\item Data-driven decision making.
\item Data mining.
\item Numerical simulations.
\item Dark matter.


\end{itemize}

\section*{Education}
\noindent
\years{2015-2019}\textsc{Undergraduate physics studies}\\ {\emph{Institution}}: Departamento de Física, Universidad de los Andes. {\emph{Dissertation}}: Simulating a collisional dark matter fluid using a Lattice-Boltzmann method. {\emph{Advisor}}: Dr. Jaime Forero.\\

\years{2019-2020}\textsc{Master in Sciences-Physics}\\ {\emph{Institution}}: Departamento de Física, Universidad de los Andes. {\emph{Dissertation}}: Searching for extragalactic variable stars using Machine Learning algorithms. {\emph{Advisor}}: Dr. Alejandro García.\\
\section*{Participation in events}
\years{2019}{MOCa 2019: Dark Matter in Colombia (Materia Oscura en Colombia).}\\ {\emph{Institution}}: Departamento de Física, Universidad de los Andes. {\emph{Talk}}: Simulating collisional dark matter.\\
\years{2019}{COCOA 2019 Medellín: VI Colombian Congress of Astronomy and Astrophysics (VI Congreso Colombiano de Astronomía y Astrofísica).}\\ {\emph{Organizers}}: Universidad de Antioquia, Parque Explora – Planetario de Medellín, Instituto Tecnológico Metropolitano ITM y Sociedad Antioqueña de Astronomía SAA. {\emph{Talk}}: Simulating Collisional Dark Matter (Simulando materia oscura colisional).\\
\years{2018}{Uniandes School of Astronomy 2018 (Escuela de Astronomía Uniandes 2018).}\\ {\emph{Institution}}: Departamento de Física, Universidad de los Andes. \\
\years{2018}{MOCa 2018: Dark Matter in Colombia (Materia Oscura en Colombia).}\\ {\emph{Institution}}: Departamento de Física, Universidad de los Andes. {\emph{Talk}}: Simulating Collisional Dark Matter.\\
\section*{Research activities}
\years{2019-ongoing}{Gravitational lens modeling using the 2.2-m ESO/MPG to measure $H_0$ (H0LICOW)\\ {\emph{Institution}}: Departamento de Física, Universidad de los Andes. {\emph{Director}}: Dr. Alejandro García and Dr. Frédéric Courbin.\\	
\years{2019-2020}{Search for extragalactic variable stars using Machine Learning algorithms.}\\ {\emph{Institution}}: Departamento de Física, Universidad de los Andes. {\emph{Advisor}}: Dr. Alejandro García.\\	
\years{2019}{Measurement of the rotation velocity of type B and A stars (Medición de la velocidad de rotación de estrellas tipo B y A).}\\ {\emph{Institution}}: Departamento de Física, Universidad de los Andes. {\emph{Advisor}}: Dr. Alejandro García.\\	
\years{2018-2020}{Simulating collisional dark matter using a lattice Boltzmann method.}\\ {\emph{Institución}}: Departamento de Física, Universidad de los Andes. {\emph{Advisor}}: Dr. Jaime Forero.\\	


\section*{Teaching experience}
\years{2019-2020}{Teaching assistant, Experimental physics I.}\\ {\emph{Institution}}: Departamento de Física, Universidad de los Andes.\\	
\years{2019-2020}{Teaching assistant, Experimental physics II.}\\ {\emph{Institution}}: Departamento de Física, Universidad de los Andes.


\section*{Other works and Courses}
\years{2019}{Design of the book "Las Bolsas de Basura" by Enrique Winter.}\\ {\emph{Editorial house}}: Escarabajo editorial.\\	
\years{2020}{Data-Driven Astronomy. Coursera: The University of Sydney.}\\ 
\years{2020}{Support Vector Machines with scikit-learn Coursera: Coursera Project Network.}\\ 

\section*{Awards and scholarships}
\years{2019}{Recognizance to best results. "Prueba Saber Pro 2018". Given by the Colombian Ministry of Education.}\\
\years{2019}{Teaching assistant with full scholarship for master studies in physics, given by Universidad de los Andes.\\
\years{2014}{Full scholarship for undergraduate studies "Bachilleres por Colombia, Programa Mario Galán Gómez", given by Ecopetrol.}\\
\years{2013}{Best student from the department of Santander. "Prueba Saber 11 2013". Given by the Colombian Ministry of Education.}\\ 

\section*{Professional Abilities}
\begin{itemize}
\item Teamwork.
\item Very high problem solving skills.
\item Creative.
\item Advanced knowledge of mathematics and physics.
\item Advanced knowledge of Statistics (including Bayesian) and artificial intelligence.
\item Attention to detail.
\item High capacity to work under pressure.
\item Languages: Spanish (native), English (C1) and German (A1).
\item Advanced use of Linux.
\item Very fast and good learner.
\item Design of articles and books in \LaTeX.
\item Programming languages: Python, Bash, R, Julia, C/C++, Java.
\item SQL: SQlite, PySQL, MySQL.
\item Machine learning models.
\item Neural networks in Pytorch, Tensorflow, Keras and Flux. 
\item Montecarlo methods.
\item MPI and OpenMP.
\item SSH and associated protocols.
\item Use of telescope, spectroscope and optical equipment.
\item Reduction and analysis of CCD images.
\item Data visualization (Dash, Seaborn, Matplotlib, Gnuplot).
\item Symbolic algebra with Maxima and Sympy.
\item Basic electronic and Arduino.
\item Design and implementation of computer simulations and numerical methods.
\item Additional software: Awk, Anaconda, IRAF, Git, Make, Pandas, Numpy, Scikit-learn, Optuna, Spyder, Jupyter, Vim.
\end{itemize}

\section*{References}
\begin{itemize}
\item Dr. Jose Alejandro Garcia Varela\\
Departamento de Física Universidad de los Andes.\\
Email: \href{mailto:josegarc@uniandes.edu.co}{josegarc@uniandes.edu.co}

\item Dr. Jaime Ernesto Forero Romero\\
Departamento de Física Universidad de los Andes.\\
Email: \href{mailto:je.forero@uniandes.edu.co}{je.forero@uniandes.edu.co}

\item Dr. Beatriz Eugenia Sabogal Martinez\\
Profesora Departamento de Física Universidad de los Andes.\\
Email: \href{mailto:bsabogal@uniandes.edu.co}{bsabogal@uniandes.edu.co}
\end{itemize}



\end{document}
\documentclass[11pt, a4paper]{article}
%\usepackage{fontspec} 
\usepackage{etaremune}

% DOCUMENT LAYOUT
\usepackage{geometry} 
\geometry{a4paper, textwidth=6.1in, textheight=10.2in, marginparsep=7pt, marginparwidth=.6in}
\setlength\parindent{0in}

\usepackage[spanish]{babel}
\usepackage[bookmarks, colorlinks, breaklinks, 
% ---- FILL IN HERE THE TITLE AND AUTHOR
	pdftitle={Javier Acevedo - vita},
	pdfauthor={Javier Acevedo},
]{hyperref}

\usepackage[utf8]{inputenc} % Acepta caracteres en castellano
\usepackage[T1]{fontenc} % Encoding de salida al pdf

\usepackage{marginnote}
\newcommand{\years}[1]{\marginnote{\scriptsize #1}}
\renewcommand*{\raggedleftmarginnote}{}
\setlength{\marginparsep}{7pt}


% ---- CUSTOM COMMANDS

\newcommand{\html}[1]{\href{#1}{\scriptsize\textsc{[html]}}}
\newcommand{\pdf}[1]{\href{#1}{\scriptsize\textsc{[pdf]}}}
\newcommand{\doi}[1]{\href{#1}{\scriptsize\textsc{[doi]}}}

\renewcommand*{\raggedleftmarginnote}{}
\setlength{\marginparsep}{7pt}
\reversemarginpar









\begin{document}
\begin{center}{\huge \bf Javier Alejandro Acevedo Barroso}\\[1cm]\end{center}
\begin{minipage}[t]{0.465\textwidth}
  Teléfono: (+57) 301-680-9844 \\
  Email: \href{mailto:ja.acevedo12@uniandes.edu.co}{ja.acevedo12@uniandes.edu.co}\\
  Email: \href{mailto:ja.acevedob12@gmail.com}{ja.acevedob12@gmail.com}\\
\end{minipage}

\hrule

\section*{Información Personal}
Nacido en Bucaramanga, Colombia, el 4 de enero de 1997 (22 años).

\section*{Áreas de Interés}
\begin{itemize}
\item Reconstrucción y modelamiento de lentes gravitacionales fuertes.
\item Búsqueda de estrellas variables.
\item Medición de distancia extragaláctica.
\item Machine Learning aplicado en astronomía
\item Simulaciones numéricas.
\item Materia oscura.
\end{itemize}

\section*{Educación}
\noindent
\years{2015-2019}\textsc{Pregrado en Física}\\ {\emph{Institución}}: Departamento de Física, Universidad de los Andes. {\emph{Tesis}}: Simulación de materia oscura colisional con un método de Lattice-Boltzmann. {\emph{Director}}: Dr. Jaime Forero.\\

\years{2019-2020}\textsc{Maestría en Ciencias-Física\footnote{Fecha tentativa de grado}}\\ {\emph{Institución}}: Departamento de Física, Universidad de los Andes. {\emph{Tesis}}: Búsqueda de estrellas variables extragalácticas usando algoritmos de Machine Learning. {\emph{Director}}: Dr. Alejandro García.\\
\section*{Participación en Eventos}
\years{2018}{Escuela de Astronomía Uniandes 2018.}\\ {\emph{Institución}}: Departamento de Física, Universidad de los Andes. \\
\years{2018}{MOCa 2018: Materia Oscura en Colombia}\\ {\emph{Institución}}: Departamento de Física, Universidad de los Andes. {\emph{Charla}}: Simulating Collisional Dark Matter.\\
\years{2019}{MOCa 2019: Materia Oscura en Colombia}\\ {\emph{Institución}}: Departamento de Física, Universidad de los Andes. {\emph{Charla}}: Simulating collisional dark matter.\\
\years{2019}{COCOA 2019 Medellín: VI Congreso Colombiano de Astronomía y Astrofísica }\\ {\emph{Organizadores}}: Universidad de Antioquia, Parque Explora – Planetario de Medellín, Instituto Tecnológico Metropolitano ITM y Sociedad Antioqueña de Astronomía SAA. {\emph{Charla}}: Simulando materia oscura colisional.\\
\section*{Actividad de Investigación}
\years{2019-2020}{Búsqueda de estrellas variables extragalácticas usando algoritmos de Machine Learning}\\ {\emph{Institución}}: Departamento de Física, Universidad de los Andes. {\emph{Director}}: Dr. Alejandro García.\\	
\years{2019}{Medición de la velocidad de rotación de estrellas tipo B y A}\\ {\emph{Institución}}: Departamento de Física, Universidad de los Andes. {\emph{Director}}: Dr. Alejandro García.\\	
\years{2018-2020}{Simulación de materia oscura colisional con un método de Lattice-Boltzmann.}\\ {\emph{Institución}}: Departamento de Física, Universidad de los Andes. {\emph{Director}}: Dr. Jaime Forero.\\	
\years{2017}{Caracterización de materiales utilizando tomografía de Muones}\\ {\emph{Institución}}: Departamento de Física, Universidad de los Andes. {\emph{Director}}: Dr. Carlos Ávila.\\

\section*{Experiencia Docente}
\years{2019}{Asistente graduado, Física experimental I.}\\ {\emph{Institución}}: Departamento de Física, Universidad de los Andes. {\emph{Supervisor}}: Germán Andrés Moreno Cely.\\	
\years{2019}{Asistente graduado, Física experimental II.}\\ {\emph{Institución}}: Departamento de Física, Universidad de los Andes. {\emph{Supervisor}}: Germán Andrés Moreno Cely.\\	
\years{2017-2018}{Tutor de la Clínica de Problemas de Física.}\\ {\emph{Institución}}: Departamento de Física, Universidad de los Andes. {\emph{Supervisor}}: Juan Diego Arango Montoya.\\	

\section*{Otro trabajo}
\years{2019}{Diagramación del libro Las Bolsas de Basura de Enrique Winter.}\\ {\emph{Editorial}}: Escarabajo editorial.\\	

\section*{Reconocimientos y Becas}
\years{2019}{Asistencia graduada para maestría en Ciencias-Física con beca completa, Universidad de los Andes.\\
\years{2014}{Beca Bachilleres por Colombia, Programa Mario Galán Gómez }. Otorgada por Ecopetrol.\\
\years{2013}{Mejor estudiande del departamento de Santander. Prueba Saber 11 2013. Otorgado por el Ministerio de Educación.}\\ 

\section*{Habilidades Adicionales}
\begin{itemize}
\item Alta capacidad de trabajo bajo presión y trabajo en equipo.
\item Lenguajes: Español (nativo), Inglés (C1), Alemán (A1).
\item Sistemas operativos Linux y Windows. 
\item Diagramación profesional en \LaTeX.
\item Lenguajes de programación: C, C++, Python, R,  Java y Bash.
\item Habilidades de ofimática.
\item Manejo de telescopio y reducción de datos astronómicos.
\item Uso de software adicional: Anaconda, IRAF, Lenstronomy, Sympy, Pytorch, Maxima.
\end{itemize}


\end{document}
\documentclass[11pt, a4paper]{article}
%\usepackage{fontspec} 
\usepackage{etaremune}

% DOCUMENT LAYOUT
\usepackage{geometry} 
\geometry{a4paper, textwidth=6.1in, textheight=10.2in, marginparsep=7pt, marginparwidth=.6in}
\setlength\parindent{0in}

\usepackage[english]{babel}
\usepackage[bookmarks, colorlinks, breaklinks, 
% ---- FILL IN HERE THE TITLE AND AUTHOR
	pdftitle={Javier Acevedo Barroso- CV},
	pdfauthor={Javier Acevedo Barroso},
]{hyperref}

\usepackage[utf8]{inputenc} % Acepta caracteres en castellano
\usepackage[T1]{fontenc} % Encoding de salida al pdf

\usepackage{marginnote}
\newcommand{\years}[1]{\marginnote{\scriptsize #1}}
\renewcommand*{\raggedleftmarginnote}{}
\setlength{\marginparsep}{7pt}

\usepackage{titlesec}

% ---- CUSTOM COMMANDS

\newcommand{\html}[1]{\href{#1}{\scriptsize\textsc{[html]}}}
\newcommand{\pdf}[1]{\href{#1}{\scriptsize\textsc{[pdf]}}}
\newcommand{\doi}[1]{\href{#1}{\scriptsize\textsc{[doi]}}}

\renewcommand*{\raggedleftmarginnote}{}
\setlength{\marginparsep}{7pt}
\reversemarginpar

% ---- REDUCING WHITESPACE AFTER SECTION

\titlespacing*{\section}
{0pt}{0.2cm plus 0.05cm minus .02cm}{0.1cm plus .1cm}
%{0pt}{1.5ex plus 1ex minus .2ex}{1.3ex plus .2ex}







\begin{document}
\begin{center}
{\huge \bf Javier Alejandro Acevedo Barroso}\\
{\Large Doctoral assistant}

\end{center}
\begin{minipage}[t]{0.59\textwidth}
  Email: \href{mailto:ja.acevedob12@gmail.com}{ja.acevedob12@gmail.com}\\
  Email: \href{mailto:javier.acevedobarroso@epfl.ch}{javier.acevedobarroso@epfl.ch}\\
  Github: \href{https://github.com/ClarkGuilty}{ClarkGuilty}\\
 
\end{minipage}
\begin{minipage}[t]{0.39\textwidth}
  Office\hfill (+41) 21-693-2402\phantom{00}  \\
  Personal\hfill (+41) 77-213-3802\phantom{00}  \\
  ORCID \hfill \href{https://orcid.org/0000-0002-9654-1711}{0000-0002-9654-1711}
\end{minipage}


\hrule

\section*{Education}
\noindent
\years{2021-2025}\textsc{Doctor in Physics\footnote{Defence scheduled for July, 2025}}\\ {Laboratory of Astrophysics, École polytechnique fédérale de Lausanne (EPFL).\newline
Searching for gravitational lenses at the dawn of the new generation of wide-field surveys.\\{\emph{Advisor}}: Dr. Frédéric Courbin.\\[-0.4cm]

\years{2019-2021}\textsc{Master in Sciences-Physics}\\ {Departamento de Física, Universidad de los Andes, Bogotá.\newline
Searching for extragalactic variable stars using Machine Learning algorithms.\\{\emph{Advisor}}: Dr. Alejandro García.\\[-0.4cm]


\years{2015-2019}\textsc{Undergraduate physics studies}\\ Departamento de Física, Universidad de los Andes, Bogotá.\\Simulating a collisional dark matter fluid using a Lattice-Boltzmann method.\\{\emph{Advisor}}: Dr. Jaime Forero.%\\[-0.9cm]

\section*{Selected publications}
\noindent
\years{2024}{Euclid: The Early Release Observations Lens Search Experiment}\hfill \href{https://arxiv.org/abs/2408.06217}{2024arXiv240806217A}\newline
\textbf{Acevedo Barroso, J. A.}, O’Riordan, C. M., Clément, B., et al.\\[-0.4cm]

\years{2024}{ Strong lensing by edge-on galaxies in UNIONS }\hfill \href{https://ui.adsabs.harvard.edu/abs/2024IAUS..381...17A}{2024IAUS..381...17A\phantom{}}\newline
\textbf{Acevedo Barroso, J. A.}, Clément, B., Courbin, F., et al.\\[-0.4cm]

\years{2024}{ Euclid. I. Overview of the Euclid mission  }\hfill \href{https://ui.adsabs.harvard.edu/abs/2024arXiv240513491E/}{2024arXiv240513491E\phantom{}}\newline
Euclid Collaboration: Mellier, Y., Abdurro’uf, \textbf{Acevedo Barroso, J. A.}, et al.\\[-0.4cm]

\years{2024}{ Euclid: Searches for strong gravitational lenses using convolutional neural nets in Early Release Observations of the Perseus field}\hfill \href{https://ui.adsabs.harvard.edu/abs/2024arXiv241116808P/abstract}{2024arXiv241116808P\phantom{}}\newline
Pearce-Casey, R., [...] , \textbf{Acevedo Barroso, J. A.},   et al.%\\[-0.9cm]

\years{2023}{The impact of human expert visual inspection on the discovery\newline of strong gravitational lenses }\hfill \href{https://ui.adsabs.harvard.edu/abs/2023MNRAS.523.4413R}{2023MNRAS.523.4413R\phantom{}}\newline
Rojas, K., [...] , \textbf{Acevedo Barroso, J. A.},   et al.%\\[-0.9cm]


\section*{Participation in events}
\years{2024}{COCOA 2024 Bucaramanga: VIII Colombian Congress of Astronomy and Astrophysics}\newline
{\emph{Talk}}: Galaxy-scale lensing in Euclid \\
\years{2024}{Third Julio Garavito Meeting}\newline
{\emph{Talk}}: Searching for lensing by edge-on galaxies in UNIONS\\
\years{2024}{Euclid Consortium Meeting 2024}\newline
{\emph{Talk}}: The ERO Lens Search Experiment\\
\years{2023}{IAU Symposia 381: Strong gravitational lensing in the era of Big Data}\\ {\emph{Talk}}: Searching for lensing by edge-on galaxies in UNIONS\\
\years{2022}{Lensing Odyssey 2022}\newline
{\emph{Talk}}: Automated lens finding: results and prospects\\
\years{2019}{COCOA 2019 Medellín: VI Colombian Congress of Astronomy and Astrophysics}\\ {\emph{Talk}}: Simulating Collisional Dark Matter (Simulando materia oscura colisional).%\\[-0.9cm]

\section*{Awards and scholarships}
\years{2019}{Recognition for Best Scores. ``Prueba Saber Pro 2018'', awarded by the Colombian Ministry of National Education}\newline
\years{2019}{Teaching assistant with full scholarship for master studies in physics, awarded by Universidad de los Andes, Bogotá}\\
\years{2014}{Full scholarship for undergraduate studies, ``Bachilleres por Colombia, Programa Mario Galán Gómez'', awarded by Ecopetrol}\\
\years{2013}{Best student of the Department of Santander. Prueba Saber 11 2013, awarded by the Colombian Ministry of National Education}\\ 

\section*{Teaching and Technical Experience}
\years{2022-2024}{Service observations for 1.2\,m Swiss National Telescope Leonard Euler: 20 nights}\newline
\years{2023}{Observations on the 3.58\,m New Technology Telescope in la Silla: 3 nights}\newline
\years{2023}{Observations on the 2.56\,m Nordic Optical Telescope in la Palma: 2 nights}\newline
\years{2022-2024}{Teaching assistant. Practical Works IV. École polytechnique fédérale de Lausanne (EPFL)}\newline
\years{2022-2023}{Teaching assistant. Astrophysics 101. École polytechnique fédérale de Lausanne (EPFL)}\newline
\years{2019-2021}{Teaching assistant. Experimental Physics I and II. Universidad de los Andes, Bogotá}
\end{document}


\section*{Interest areas}
\begin{itemize}
%\item Reconstrucción y modelamiento de lentes gravitacionales fuertes.
\item Reconstruction and modelling of gravitational lens.
\item Search for stellar variability.
\item Local extragalactic distance measurement
\item Machine Learning in astronomy.
\item Numerical simulations.
\item Dark matter.
\end{itemize}

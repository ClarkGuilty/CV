\documentclass[11pt, a4paper]{article}
%\usepackage{fontspec} 
\usepackage{etaremune}

% DOCUMENT LAYOUT
\usepackage{geometry} 
\geometry{a4paper, textwidth=6.1in, textheight=10.2in, marginparsep=7pt, marginparwidth=.6in}
\setlength\parindent{0in}

\usepackage[english]{babel}
\usepackage[bookmarks, colorlinks, breaklinks, 
% ---- FILL IN HERE THE TITLE AND AUTHOR
	pdftitle={Javier Acevedo - vita},
	pdfauthor={Javier Acevedo},
]{hyperref}

\usepackage[utf8]{inputenc} % Acepta caracteres en castellano
\usepackage[T1]{fontenc} % Encoding de salida al pdf

\usepackage{marginnote}
\newcommand{\years}[1]{\marginnote{\scriptsize #1}}
\renewcommand*{\raggedleftmarginnote}{}
\setlength{\marginparsep}{7pt}


% ---- CUSTOM COMMANDS

\newcommand{\html}[1]{\href{#1}{\scriptsize\textsc{[html]}}}
\newcommand{\pdf}[1]{\href{#1}{\scriptsize\textsc{[pdf]}}}
\newcommand{\doi}[1]{\href{#1}{\scriptsize\textsc{[doi]}}}

\renewcommand*{\raggedleftmarginnote}{}
\setlength{\marginparsep}{7pt}
\reversemarginpar









\begin{document}
\begin{center}{\huge \bf Javier Alejandro Acevedo Barroso}\\[1cm]\end{center}
\begin{minipage}[t]{0.465\textwidth}
  Phone: (+57) 301-680-9844 \\
  Email: \href{mailto:ja.acevedo12@uniandes.edu.co}{ja.acevedo12@uniandes.edu.co}\\
  Email: \href{mailto:ja.acevedob12@gmail.com}{ja.acevedob12@gmail.com}\\
  Github: \href{https://github.com/ClarkGuilty}{ClarkGuilty}\\
\end{minipage}

\hrule

\section*{Personal information}
Born in Bucaramanga, Colombia, on January 4\textsuperscript{th} 1997 (22 years).

\section*{Interest areas}
\begin{itemize}
%\item Reconstrucción y modelamiento de lentes gravitacionales fuertes.
\item Reconstruction and modelling of gravitational lens.
\item Search for stellar variability.
\item Local extragalactic distance measurement
\item Machine Learning in astronomy.
\item Numerical simulations.
\item Dark matter.
\end{itemize}

\section*{Education}
\noindent
\years{2015-2019}\textsc{Undergraduate physics studies}\\ {\emph{Institution}}: Departamento de Física, Universidad de los Andes. {\emph{Dissertation}}: Simulating a collisional dark matter fluid using a Lattice-Boltzmann method. {\emph{Advisor}}: Dr. Jaime Forero.\\

\years{2019-2020}\textsc{Master in Sciences-Physics\footnote{Tentative graduation date}}\\ {\emph{Institution}}: Departamento de Física, Universidad de los Andes. {\emph{Dissertation}}: Search for extragalactic variable stars using Machine Learning algorithms. {\emph{Advisor}}: Dr. Alejandro García.\\
\section*{Participation in events}
\years{2019}{MOCa 2019: Dark Matter in Colombia (Materia Oscura en Colombia).}\\ {\emph{Institution}}: Departamento de Física, Universidad de los Andes. {\emph{Talk}}: Simulating collisional dark matter.\\
\years{2019}{COCOA 2019 Medellín: VI Colombian Congress of Astronomy and Astrophysics (VI Congreso Colombiano de Astronomía y Astrofísica).}\\ {\emph{Organizers}}: Universidad de Antioquia, Parque Explora – Planetario de Medellín, Instituto Tecnológico Metropolitano ITM y Sociedad Antioqueña de Astronomía SAA. {\emph{Talk}}: Simulating Collisional Dark Matter (Simulando materia oscura colisional).\\
\years{2018}{Uniandes School of Astronomy 2018 (Escuela de Astronomía Uniandes 2018).}\\ {\emph{Institution}}: Departamento de Física, Universidad de los Andes. \\
\years{2018}{MOCa 2018: Dark Matter in Colombia (Materia Oscura en Colombia).}\\ {\emph{Institution}}: Departamento de Física, Universidad de los Andes. {\emph{Talk}}: Simulating Collisional Dark Matter.\\
\section*{Research activities}
\years{2019-2020}{Search for extragalactic variable stars using Machine Learning algorithms (Búsqueda de estrellas variables extragalácticas usando algoritmos de Machine Learning).}\\ {\emph{Institution}}: Departamento de Física, Universidad de los Andes. {\emph{Advisor}}: Dr. Alejandro García.\\	
\years{2019}{Measurement of the rotation velocity of type B and A stars (Medición de la velocidad de rotación de estrellas tipo B y A).}\\ {\emph{Institution}}: Departamento de Física, Universidad de los Andes. {\emph{Advisor}}: Dr. Alejandro García.\\	
\years{2018-2020}{Simulating collisional dark matter using a lattice Boltzmann method.}\\ {\emph{Institución}}: Departamento de Física, Universidad de los Andes. {\emph{Advisor}}: Dr. Jaime Forero.\\	


\section*{Teaching experience}
\years{2019}{Teaching assistant, Experimental physics I.}\\ {\emph{Institution}}: Departamento de Física, Universidad de los Andes.\\	
\years{2019}{Teaching assistant, Experimental physics II.}\\ {\emph{Institution}}: Departamento de Física, Universidad de los Andes.


\section*{Other works}
\years{2019}{Design of the book "Las Bolsas de Basura" by Enrique Winter.}\\ {\emph{Editorial house}}: Escarabajo editorial.\\	

\section*{Awards and scholarships}
\years{2019}{Teaching assistant with full scholarship for master studies in physics, given by Universidad de los Andes.\\
\years{2014}{Full scholarship for undergraduate studies "Bachilleres por Colombia, Programa Mario Galán Gómez", given by Ecopetrol.}\\
\years{2013}{Best student from Santander department. "Prueba Saber 11 2013". given by the Colombian Ministry of Education.}\\ 

\section*{Professional Abilities}
\begin{itemize}
\item Teamwork, specially in computational tasks.
\item High capacity of working under pressure.
\item Languages: Spanish (native), English (C1), German (A1).
\item Advanced use of Linux.
\item Design of articles and books in \LaTeX.
\item Programming languages: Python (advanced), Bash (intermediate), C (intermediate), R (basic), Java (basic), C++ (basic).
\item Reduction and analysis of CCD images.
\item Additional software: Anaconda, IRAF, Git, Make, Pandas, Numpy, Pytorch, Sympy and Spyder.
\end{itemize}


\end{document}